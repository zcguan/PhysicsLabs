
% Default to the notebook output style

    


% Inherit from the specified cell style.




    
\documentclass[11pt]{article}

    
    
    \usepackage[T1]{fontenc}
    % Nicer default font (+ math font) than Computer Modern for most use cases
    \usepackage{mathpazo}

    % Basic figure setup, for now with no caption control since it's done
    % automatically by Pandoc (which extracts ![](path) syntax from Markdown).
    \usepackage{graphicx}
    % We will generate all images so they have a width \maxwidth. This means
    % that they will get their normal width if they fit onto the page, but
    % are scaled down if they would overflow the margins.
    \makeatletter
    \def\maxwidth{\ifdim\Gin@nat@width>\linewidth\linewidth
    \else\Gin@nat@width\fi}
    \makeatother
    \let\Oldincludegraphics\includegraphics
    % Set max figure width to be 80% of text width, for now hardcoded.
    \renewcommand{\includegraphics}[1]{\Oldincludegraphics[width=.8\maxwidth]{#1}}
    % Ensure that by default, figures have no caption (until we provide a
    % proper Figure object with a Caption API and a way to capture that
    % in the conversion process - todo).
    \usepackage{caption}
    \DeclareCaptionLabelFormat{nolabel}{}
    \captionsetup{labelformat=nolabel}

    \usepackage{adjustbox} % Used to constrain images to a maximum size 
    \usepackage{xcolor} % Allow colors to be defined
    \usepackage{enumerate} % Needed for markdown enumerations to work
    \usepackage{geometry} % Used to adjust the document margins
    \usepackage{amsmath} % Equations
    \usepackage{amssymb} % Equations
    \usepackage{textcomp} % defines textquotesingle
    % Hack from http://tex.stackexchange.com/a/47451/13684:
    \AtBeginDocument{%
        \def\PYZsq{\textquotesingle}% Upright quotes in Pygmentized code
    }
    \usepackage{upquote} % Upright quotes for verbatim code
    \usepackage{eurosym} % defines \euro
    \usepackage[mathletters]{ucs} % Extended unicode (utf-8) support
    \usepackage[utf8x]{inputenc} % Allow utf-8 characters in the tex document
    \usepackage{fancyvrb} % verbatim replacement that allows latex
    \usepackage{grffile} % extends the file name processing of package graphics 
                         % to support a larger range 
    % The hyperref package gives us a pdf with properly built
    % internal navigation ('pdf bookmarks' for the table of contents,
    % internal cross-reference links, web links for URLs, etc.)
    \usepackage{hyperref}
    \usepackage{longtable} % longtable support required by pandoc >1.10
    \usepackage{booktabs}  % table support for pandoc > 1.12.2
    \usepackage[inline]{enumitem} % IRkernel/repr support (it uses the enumerate* environment)
    \usepackage[normalem]{ulem} % ulem is needed to support strikethroughs (\sout)
                                % normalem makes italics be italics, not underlines
    

    
    
    % Colors for the hyperref package
    \definecolor{urlcolor}{rgb}{0,.145,.698}
    \definecolor{linkcolor}{rgb}{.71,0.21,0.01}
    \definecolor{citecolor}{rgb}{.12,.54,.11}

    % ANSI colors
    \definecolor{ansi-black}{HTML}{3E424D}
    \definecolor{ansi-black-intense}{HTML}{282C36}
    \definecolor{ansi-red}{HTML}{E75C58}
    \definecolor{ansi-red-intense}{HTML}{B22B31}
    \definecolor{ansi-green}{HTML}{00A250}
    \definecolor{ansi-green-intense}{HTML}{007427}
    \definecolor{ansi-yellow}{HTML}{DDB62B}
    \definecolor{ansi-yellow-intense}{HTML}{B27D12}
    \definecolor{ansi-blue}{HTML}{208FFB}
    \definecolor{ansi-blue-intense}{HTML}{0065CA}
    \definecolor{ansi-magenta}{HTML}{D160C4}
    \definecolor{ansi-magenta-intense}{HTML}{A03196}
    \definecolor{ansi-cyan}{HTML}{60C6C8}
    \definecolor{ansi-cyan-intense}{HTML}{258F8F}
    \definecolor{ansi-white}{HTML}{C5C1B4}
    \definecolor{ansi-white-intense}{HTML}{A1A6B2}

    % commands and environments needed by pandoc snippets
    % extracted from the output of `pandoc -s`
    \providecommand{\tightlist}{%
      \setlength{\itemsep}{0pt}\setlength{\parskip}{0pt}}
    \DefineVerbatimEnvironment{Highlighting}{Verbatim}{commandchars=\\\{\}}
    % Add ',fontsize=\small' for more characters per line
    \newenvironment{Shaded}{}{}
    \newcommand{\KeywordTok}[1]{\textcolor[rgb]{0.00,0.44,0.13}{\textbf{{#1}}}}
    \newcommand{\DataTypeTok}[1]{\textcolor[rgb]{0.56,0.13,0.00}{{#1}}}
    \newcommand{\DecValTok}[1]{\textcolor[rgb]{0.25,0.63,0.44}{{#1}}}
    \newcommand{\BaseNTok}[1]{\textcolor[rgb]{0.25,0.63,0.44}{{#1}}}
    \newcommand{\FloatTok}[1]{\textcolor[rgb]{0.25,0.63,0.44}{{#1}}}
    \newcommand{\CharTok}[1]{\textcolor[rgb]{0.25,0.44,0.63}{{#1}}}
    \newcommand{\StringTok}[1]{\textcolor[rgb]{0.25,0.44,0.63}{{#1}}}
    \newcommand{\CommentTok}[1]{\textcolor[rgb]{0.38,0.63,0.69}{\textit{{#1}}}}
    \newcommand{\OtherTok}[1]{\textcolor[rgb]{0.00,0.44,0.13}{{#1}}}
    \newcommand{\AlertTok}[1]{\textcolor[rgb]{1.00,0.00,0.00}{\textbf{{#1}}}}
    \newcommand{\FunctionTok}[1]{\textcolor[rgb]{0.02,0.16,0.49}{{#1}}}
    \newcommand{\RegionMarkerTok}[1]{{#1}}
    \newcommand{\ErrorTok}[1]{\textcolor[rgb]{1.00,0.00,0.00}{\textbf{{#1}}}}
    \newcommand{\NormalTok}[1]{{#1}}
    
    % Additional commands for more recent versions of Pandoc
    \newcommand{\ConstantTok}[1]{\textcolor[rgb]{0.53,0.00,0.00}{{#1}}}
    \newcommand{\SpecialCharTok}[1]{\textcolor[rgb]{0.25,0.44,0.63}{{#1}}}
    \newcommand{\VerbatimStringTok}[1]{\textcolor[rgb]{0.25,0.44,0.63}{{#1}}}
    \newcommand{\SpecialStringTok}[1]{\textcolor[rgb]{0.73,0.40,0.53}{{#1}}}
    \newcommand{\ImportTok}[1]{{#1}}
    \newcommand{\DocumentationTok}[1]{\textcolor[rgb]{0.73,0.13,0.13}{\textit{{#1}}}}
    \newcommand{\AnnotationTok}[1]{\textcolor[rgb]{0.38,0.63,0.69}{\textbf{\textit{{#1}}}}}
    \newcommand{\CommentVarTok}[1]{\textcolor[rgb]{0.38,0.63,0.69}{\textbf{\textit{{#1}}}}}
    \newcommand{\VariableTok}[1]{\textcolor[rgb]{0.10,0.09,0.49}{{#1}}}
    \newcommand{\ControlFlowTok}[1]{\textcolor[rgb]{0.00,0.44,0.13}{\textbf{{#1}}}}
    \newcommand{\OperatorTok}[1]{\textcolor[rgb]{0.40,0.40,0.40}{{#1}}}
    \newcommand{\BuiltInTok}[1]{{#1}}
    \newcommand{\ExtensionTok}[1]{{#1}}
    \newcommand{\PreprocessorTok}[1]{\textcolor[rgb]{0.74,0.48,0.00}{{#1}}}
    \newcommand{\AttributeTok}[1]{\textcolor[rgb]{0.49,0.56,0.16}{{#1}}}
    \newcommand{\InformationTok}[1]{\textcolor[rgb]{0.38,0.63,0.69}{\textbf{\textit{{#1}}}}}
    \newcommand{\WarningTok}[1]{\textcolor[rgb]{0.38,0.63,0.69}{\textbf{\textit{{#1}}}}}
    
    
    % Define a nice break command that doesn't care if a line doesn't already
    % exist.
    \def\br{\hspace*{\fill} \\* }
    % Math Jax compatability definitions
    \def\gt{>}
    \def\lt{<}
    % Document parameters
    \title{Lab2-1}
    
    
    

    % Pygments definitions
    
\makeatletter
\def\PY@reset{\let\PY@it=\relax \let\PY@bf=\relax%
    \let\PY@ul=\relax \let\PY@tc=\relax%
    \let\PY@bc=\relax \let\PY@ff=\relax}
\def\PY@tok#1{\csname PY@tok@#1\endcsname}
\def\PY@toks#1+{\ifx\relax#1\empty\else%
    \PY@tok{#1}\expandafter\PY@toks\fi}
\def\PY@do#1{\PY@bc{\PY@tc{\PY@ul{%
    \PY@it{\PY@bf{\PY@ff{#1}}}}}}}
\def\PY#1#2{\PY@reset\PY@toks#1+\relax+\PY@do{#2}}

\expandafter\def\csname PY@tok@w\endcsname{\def\PY@tc##1{\textcolor[rgb]{0.73,0.73,0.73}{##1}}}
\expandafter\def\csname PY@tok@c\endcsname{\let\PY@it=\textit\def\PY@tc##1{\textcolor[rgb]{0.25,0.50,0.50}{##1}}}
\expandafter\def\csname PY@tok@cp\endcsname{\def\PY@tc##1{\textcolor[rgb]{0.74,0.48,0.00}{##1}}}
\expandafter\def\csname PY@tok@k\endcsname{\let\PY@bf=\textbf\def\PY@tc##1{\textcolor[rgb]{0.00,0.50,0.00}{##1}}}
\expandafter\def\csname PY@tok@kp\endcsname{\def\PY@tc##1{\textcolor[rgb]{0.00,0.50,0.00}{##1}}}
\expandafter\def\csname PY@tok@kt\endcsname{\def\PY@tc##1{\textcolor[rgb]{0.69,0.00,0.25}{##1}}}
\expandafter\def\csname PY@tok@o\endcsname{\def\PY@tc##1{\textcolor[rgb]{0.40,0.40,0.40}{##1}}}
\expandafter\def\csname PY@tok@ow\endcsname{\let\PY@bf=\textbf\def\PY@tc##1{\textcolor[rgb]{0.67,0.13,1.00}{##1}}}
\expandafter\def\csname PY@tok@nb\endcsname{\def\PY@tc##1{\textcolor[rgb]{0.00,0.50,0.00}{##1}}}
\expandafter\def\csname PY@tok@nf\endcsname{\def\PY@tc##1{\textcolor[rgb]{0.00,0.00,1.00}{##1}}}
\expandafter\def\csname PY@tok@nc\endcsname{\let\PY@bf=\textbf\def\PY@tc##1{\textcolor[rgb]{0.00,0.00,1.00}{##1}}}
\expandafter\def\csname PY@tok@nn\endcsname{\let\PY@bf=\textbf\def\PY@tc##1{\textcolor[rgb]{0.00,0.00,1.00}{##1}}}
\expandafter\def\csname PY@tok@ne\endcsname{\let\PY@bf=\textbf\def\PY@tc##1{\textcolor[rgb]{0.82,0.25,0.23}{##1}}}
\expandafter\def\csname PY@tok@nv\endcsname{\def\PY@tc##1{\textcolor[rgb]{0.10,0.09,0.49}{##1}}}
\expandafter\def\csname PY@tok@no\endcsname{\def\PY@tc##1{\textcolor[rgb]{0.53,0.00,0.00}{##1}}}
\expandafter\def\csname PY@tok@nl\endcsname{\def\PY@tc##1{\textcolor[rgb]{0.63,0.63,0.00}{##1}}}
\expandafter\def\csname PY@tok@ni\endcsname{\let\PY@bf=\textbf\def\PY@tc##1{\textcolor[rgb]{0.60,0.60,0.60}{##1}}}
\expandafter\def\csname PY@tok@na\endcsname{\def\PY@tc##1{\textcolor[rgb]{0.49,0.56,0.16}{##1}}}
\expandafter\def\csname PY@tok@nt\endcsname{\let\PY@bf=\textbf\def\PY@tc##1{\textcolor[rgb]{0.00,0.50,0.00}{##1}}}
\expandafter\def\csname PY@tok@nd\endcsname{\def\PY@tc##1{\textcolor[rgb]{0.67,0.13,1.00}{##1}}}
\expandafter\def\csname PY@tok@s\endcsname{\def\PY@tc##1{\textcolor[rgb]{0.73,0.13,0.13}{##1}}}
\expandafter\def\csname PY@tok@sd\endcsname{\let\PY@it=\textit\def\PY@tc##1{\textcolor[rgb]{0.73,0.13,0.13}{##1}}}
\expandafter\def\csname PY@tok@si\endcsname{\let\PY@bf=\textbf\def\PY@tc##1{\textcolor[rgb]{0.73,0.40,0.53}{##1}}}
\expandafter\def\csname PY@tok@se\endcsname{\let\PY@bf=\textbf\def\PY@tc##1{\textcolor[rgb]{0.73,0.40,0.13}{##1}}}
\expandafter\def\csname PY@tok@sr\endcsname{\def\PY@tc##1{\textcolor[rgb]{0.73,0.40,0.53}{##1}}}
\expandafter\def\csname PY@tok@ss\endcsname{\def\PY@tc##1{\textcolor[rgb]{0.10,0.09,0.49}{##1}}}
\expandafter\def\csname PY@tok@sx\endcsname{\def\PY@tc##1{\textcolor[rgb]{0.00,0.50,0.00}{##1}}}
\expandafter\def\csname PY@tok@m\endcsname{\def\PY@tc##1{\textcolor[rgb]{0.40,0.40,0.40}{##1}}}
\expandafter\def\csname PY@tok@gh\endcsname{\let\PY@bf=\textbf\def\PY@tc##1{\textcolor[rgb]{0.00,0.00,0.50}{##1}}}
\expandafter\def\csname PY@tok@gu\endcsname{\let\PY@bf=\textbf\def\PY@tc##1{\textcolor[rgb]{0.50,0.00,0.50}{##1}}}
\expandafter\def\csname PY@tok@gd\endcsname{\def\PY@tc##1{\textcolor[rgb]{0.63,0.00,0.00}{##1}}}
\expandafter\def\csname PY@tok@gi\endcsname{\def\PY@tc##1{\textcolor[rgb]{0.00,0.63,0.00}{##1}}}
\expandafter\def\csname PY@tok@gr\endcsname{\def\PY@tc##1{\textcolor[rgb]{1.00,0.00,0.00}{##1}}}
\expandafter\def\csname PY@tok@ge\endcsname{\let\PY@it=\textit}
\expandafter\def\csname PY@tok@gs\endcsname{\let\PY@bf=\textbf}
\expandafter\def\csname PY@tok@gp\endcsname{\let\PY@bf=\textbf\def\PY@tc##1{\textcolor[rgb]{0.00,0.00,0.50}{##1}}}
\expandafter\def\csname PY@tok@go\endcsname{\def\PY@tc##1{\textcolor[rgb]{0.53,0.53,0.53}{##1}}}
\expandafter\def\csname PY@tok@gt\endcsname{\def\PY@tc##1{\textcolor[rgb]{0.00,0.27,0.87}{##1}}}
\expandafter\def\csname PY@tok@err\endcsname{\def\PY@bc##1{\setlength{\fboxsep}{0pt}\fcolorbox[rgb]{1.00,0.00,0.00}{1,1,1}{\strut ##1}}}
\expandafter\def\csname PY@tok@kc\endcsname{\let\PY@bf=\textbf\def\PY@tc##1{\textcolor[rgb]{0.00,0.50,0.00}{##1}}}
\expandafter\def\csname PY@tok@kd\endcsname{\let\PY@bf=\textbf\def\PY@tc##1{\textcolor[rgb]{0.00,0.50,0.00}{##1}}}
\expandafter\def\csname PY@tok@kn\endcsname{\let\PY@bf=\textbf\def\PY@tc##1{\textcolor[rgb]{0.00,0.50,0.00}{##1}}}
\expandafter\def\csname PY@tok@kr\endcsname{\let\PY@bf=\textbf\def\PY@tc##1{\textcolor[rgb]{0.00,0.50,0.00}{##1}}}
\expandafter\def\csname PY@tok@bp\endcsname{\def\PY@tc##1{\textcolor[rgb]{0.00,0.50,0.00}{##1}}}
\expandafter\def\csname PY@tok@fm\endcsname{\def\PY@tc##1{\textcolor[rgb]{0.00,0.00,1.00}{##1}}}
\expandafter\def\csname PY@tok@vc\endcsname{\def\PY@tc##1{\textcolor[rgb]{0.10,0.09,0.49}{##1}}}
\expandafter\def\csname PY@tok@vg\endcsname{\def\PY@tc##1{\textcolor[rgb]{0.10,0.09,0.49}{##1}}}
\expandafter\def\csname PY@tok@vi\endcsname{\def\PY@tc##1{\textcolor[rgb]{0.10,0.09,0.49}{##1}}}
\expandafter\def\csname PY@tok@vm\endcsname{\def\PY@tc##1{\textcolor[rgb]{0.10,0.09,0.49}{##1}}}
\expandafter\def\csname PY@tok@sa\endcsname{\def\PY@tc##1{\textcolor[rgb]{0.73,0.13,0.13}{##1}}}
\expandafter\def\csname PY@tok@sb\endcsname{\def\PY@tc##1{\textcolor[rgb]{0.73,0.13,0.13}{##1}}}
\expandafter\def\csname PY@tok@sc\endcsname{\def\PY@tc##1{\textcolor[rgb]{0.73,0.13,0.13}{##1}}}
\expandafter\def\csname PY@tok@dl\endcsname{\def\PY@tc##1{\textcolor[rgb]{0.73,0.13,0.13}{##1}}}
\expandafter\def\csname PY@tok@s2\endcsname{\def\PY@tc##1{\textcolor[rgb]{0.73,0.13,0.13}{##1}}}
\expandafter\def\csname PY@tok@sh\endcsname{\def\PY@tc##1{\textcolor[rgb]{0.73,0.13,0.13}{##1}}}
\expandafter\def\csname PY@tok@s1\endcsname{\def\PY@tc##1{\textcolor[rgb]{0.73,0.13,0.13}{##1}}}
\expandafter\def\csname PY@tok@mb\endcsname{\def\PY@tc##1{\textcolor[rgb]{0.40,0.40,0.40}{##1}}}
\expandafter\def\csname PY@tok@mf\endcsname{\def\PY@tc##1{\textcolor[rgb]{0.40,0.40,0.40}{##1}}}
\expandafter\def\csname PY@tok@mh\endcsname{\def\PY@tc##1{\textcolor[rgb]{0.40,0.40,0.40}{##1}}}
\expandafter\def\csname PY@tok@mi\endcsname{\def\PY@tc##1{\textcolor[rgb]{0.40,0.40,0.40}{##1}}}
\expandafter\def\csname PY@tok@il\endcsname{\def\PY@tc##1{\textcolor[rgb]{0.40,0.40,0.40}{##1}}}
\expandafter\def\csname PY@tok@mo\endcsname{\def\PY@tc##1{\textcolor[rgb]{0.40,0.40,0.40}{##1}}}
\expandafter\def\csname PY@tok@ch\endcsname{\let\PY@it=\textit\def\PY@tc##1{\textcolor[rgb]{0.25,0.50,0.50}{##1}}}
\expandafter\def\csname PY@tok@cm\endcsname{\let\PY@it=\textit\def\PY@tc##1{\textcolor[rgb]{0.25,0.50,0.50}{##1}}}
\expandafter\def\csname PY@tok@cpf\endcsname{\let\PY@it=\textit\def\PY@tc##1{\textcolor[rgb]{0.25,0.50,0.50}{##1}}}
\expandafter\def\csname PY@tok@c1\endcsname{\let\PY@it=\textit\def\PY@tc##1{\textcolor[rgb]{0.25,0.50,0.50}{##1}}}
\expandafter\def\csname PY@tok@cs\endcsname{\let\PY@it=\textit\def\PY@tc##1{\textcolor[rgb]{0.25,0.50,0.50}{##1}}}

\def\PYZbs{\char`\\}
\def\PYZus{\char`\_}
\def\PYZob{\char`\{}
\def\PYZcb{\char`\}}
\def\PYZca{\char`\^}
\def\PYZam{\char`\&}
\def\PYZlt{\char`\<}
\def\PYZgt{\char`\>}
\def\PYZsh{\char`\#}
\def\PYZpc{\char`\%}
\def\PYZdl{\char`\$}
\def\PYZhy{\char`\-}
\def\PYZsq{\char`\'}
\def\PYZdq{\char`\"}
\def\PYZti{\char`\~}
% for compatibility with earlier versions
\def\PYZat{@}
\def\PYZlb{[}
\def\PYZrb{]}
\makeatother


    % Exact colors from NB
    \definecolor{incolor}{rgb}{0.0, 0.0, 0.5}
    \definecolor{outcolor}{rgb}{0.545, 0.0, 0.0}



    
    % Prevent overflowing lines due to hard-to-break entities
    \sloppy 
    % Setup hyperref package
    \hypersetup{
      breaklinks=true,  % so long urls are correctly broken across lines
      colorlinks=true,
      urlcolor=urlcolor,
      linkcolor=linkcolor,
      citecolor=citecolor,
      }
    % Slightly bigger margins than the latex defaults
    
    \geometry{verbose,tmargin=1in,bmargin=1in,lmargin=1in,rmargin=1in}
    
    

    \begin{document}
    
    
    \maketitle
    
    

    
    \begin{Verbatim}[commandchars=\\\{\}]
{\color{incolor}In [{\color{incolor}1}]:} \PY{k+kn}{import} \PY{n+nn}{numpy} \PY{k}{as} \PY{n+nn}{np}
        \PY{k+kn}{import} \PY{n+nn}{os}
        \PY{k+kn}{import} \PY{n+nn}{matplotlib}\PY{n+nn}{.}\PY{n+nn}{pyplot} \PY{k}{as} \PY{n+nn}{plot}
\end{Verbatim}


    \begin{Verbatim}[commandchars=\\\{\}]
{\color{incolor}In [{\color{incolor}2}]:} \PY{k}{def} \PY{n+nf}{data\PYZus{}generator}\PY{p}{(}\PY{n}{mean}\PY{p}{,} \PY{n}{std}\PY{p}{,} \PY{n}{n\PYZus{}measurements} \PY{o}{=} \PY{l+m+mi}{500}\PY{p}{,} \PY{n}{n\PYZus{}samples} \PY{o}{=} \PY{l+m+mi}{20}\PY{p}{)}\PY{p}{:}
            \PY{l+s+sd}{\PYZdq{}\PYZdq{}\PYZdq{}}
        \PY{l+s+sd}{Generate an array of measurements from a normal distribution.}
        \PY{l+s+sd}{Parameters}
        \PY{l+s+sd}{\PYZhy{}\PYZhy{}\PYZhy{}\PYZhy{}\PYZhy{}\PYZhy{}\PYZhy{}\PYZhy{}\PYZhy{}\PYZhy{}}
        \PY{l+s+sd}{mean : float}
        \PY{l+s+sd}{Desired mean value of the measurements.}
        \PY{l+s+sd}{std : float}
        \PY{l+s+sd}{Desired standard deviation of the measurements.}
        \PY{l+s+sd}{n\PYZus{}measurements : int, optional}
        \PY{l+s+sd}{Number of separate measurements. Default is 500 measurements.}
        \PY{l+s+sd}{n\PYZus{}samples : int, optional}
        \PY{l+s+sd}{Number of samples taken per measurement. Default is 20 sam}
        \PY{l+s+sd}{ples. }
        \PY{l+s+sd}{per measurement.}
        \PY{l+s+sd}{Returns}
        \PY{l+s+sd}{\PYZhy{}\PYZhy{}\PYZhy{}\PYZhy{}\PYZhy{}\PYZhy{}\PYZhy{}}
        \PY{l+s+sd}{data : numpy.ndarray}
        \PY{l+s+sd}{Array representing the experimental data. Each measurement }
        \PY{l+s+sd}{(composed of many samples) is a row of this array:}
        \PY{l+s+sd}{\PYZhy{}\PYZhy{}\PYZhy{}\PYZhy{}\PYZhy{}\PYZhy{}\PYZhy{}\PYZhy{}\PYZhy{}\PYZhy{}\PYZhy{}\PYZhy{}\PYZhy{}\PYZhy{}\PYZhy{}\PYZhy{}\PYZhy{}\PYZhy{}\PYZhy{}\PYZhy{}\PYZhy{}\PYZhy{}\PYZhy{}\PYZhy{}\PYZhy{}\PYZhy{}}
        \PY{l+s+sd}{\PYZhy{}\PYZhy{}\PYZhy{}\PYZhy{}\PYZhy{}\PYZhy{}\PYZhy{}\PYZhy{}\PYZhy{}}
        \PY{l+s+sd}{meas1    | sample0 | sample1 | sample2 | ...}
        \PY{l+s+sd}{meas2    | sample0 | sample1 | sample2 | ...}
        \PY{l+s+sd}{meas3    | sample0 | sample1 | sample2 | ...}
        \PY{l+s+sd}{...}
        \PY{l+s+sd}{    \PYZdq{}\PYZdq{}\PYZdq{}}
        
            \PY{k}{return} \PY{n}{np}\PY{o}{.}\PY{n}{random}\PY{o}{.}\PY{n}{normal}\PY{p}{(}\PY{n}{loc} \PY{o}{=} \PY{n}{mean}\PY{p}{,} \PY{n}{scale} \PY{o}{=} \PY{n}{std}\PY{p}{,} \PY{n}{size} \PY{o}{=} \PY{p}{(}\PY{n}{n\PYZus{}measurements}\PY{p}{,} \PY{n}{n\PYZus{}samples}\PY{p}{)}\PY{p}{)}
\end{Verbatim}


    \begin{Verbatim}[commandchars=\\\{\}]
{\color{incolor}In [{\color{incolor}3}]:} \PY{k}{def} \PY{n+nf}{Gaussian}\PY{p}{(}\PY{n}{Amp}\PY{p}{,}\PY{n}{x}\PY{p}{,} \PY{n}{u}\PY{p}{,} \PY{n}{s}\PY{p}{)}\PY{p}{:}
            \PY{k}{return} \PY{n}{Amp}\PY{o}{*}\PY{n}{np}\PY{o}{.}\PY{n}{exp}\PY{p}{(}\PY{o}{\PYZhy{}}\PY{p}{(}\PY{n}{x}\PY{o}{\PYZhy{}}\PY{n}{u}\PY{p}{)}\PY{o}{*}\PY{p}{(}\PY{n}{x}\PY{o}{\PYZhy{}}\PY{n}{u}\PY{p}{)}\PY{o}{/}\PY{p}{(}\PY{l+m+mi}{2}\PY{o}{*}\PY{n}{s}\PY{o}{*}\PY{n}{s}\PY{p}{)}\PY{p}{)}\PY{o}{/}\PY{p}{(}\PY{n}{s}\PY{o}{*}\PY{n}{np}\PY{o}{.}\PY{n}{sqrt}\PY{p}{(}\PY{l+m+mi}{2}\PY{o}{*}\PY{n}{np}\PY{o}{.}\PY{n}{pi}\PY{p}{)}\PY{p}{)}
\end{Verbatim}


    \begin{Verbatim}[commandchars=\\\{\}]
{\color{incolor}In [{\color{incolor}4}]:} \PY{n}{data} \PY{o}{=} \PY{l+s+s2}{\PYZdq{}}\PY{l+s+s2}{rollingcylinderdata.csv}\PY{l+s+s2}{\PYZdq{}}
        \PY{n}{result} \PY{o}{=} \PY{n}{np}\PY{o}{.}\PY{n}{transpose}\PY{p}{(}\PY{n}{np}\PY{o}{.}\PY{n}{loadtxt}\PY{p}{(}\PY{n}{data}\PY{p}{,}\PY{n}{delimiter}\PY{o}{=}\PY{l+s+s1}{\PYZsq{}}\PY{l+s+s1}{,}\PY{l+s+s1}{\PYZsq{}}\PY{p}{)}\PY{p}{)}
\end{Verbatim}


    \begin{Verbatim}[commandchars=\\\{\}]
{\color{incolor}In [{\color{incolor}5}]:} \PY{n}{mean} \PY{o}{=} \PY{n}{np}\PY{o}{.}\PY{n}{mean}\PY{p}{(}\PY{n}{result}\PY{p}{[}\PY{l+m+mi}{3}\PY{p}{]}\PY{p}{)}
        \PY{n}{std} \PY{o}{=} \PY{n}{np}\PY{o}{.}\PY{n}{std}\PY{p}{(}\PY{n}{result}\PY{p}{[}\PY{l+m+mi}{3}\PY{p}{]}\PY{p}{)}
\end{Verbatim}


    \begin{Verbatim}[commandchars=\\\{\}]
{\color{incolor}In [{\color{incolor}40}]:} \PY{n}{sim} \PY{o}{=} \PY{n}{data\PYZus{}generator}\PY{p}{(}\PY{n}{mean}\PY{p}{,}\PY{n}{std}\PY{p}{)}
         \PY{n}{simdata} \PY{o}{=} \PY{n}{np}\PY{o}{.}\PY{n}{ravel}\PY{p}{(}\PY{n}{sim}\PY{p}{)}
\end{Verbatim}


    \begin{Verbatim}[commandchars=\\\{\}]
{\color{incolor}In [{\color{incolor}7}]:} \PY{c+c1}{\PYZsh{} mean and std of the distribution of measurements}
        \PY{n}{meansim} \PY{o}{=} \PY{n}{np}\PY{o}{.}\PY{n}{mean}\PY{p}{(}\PY{n}{simdata}\PY{p}{)}
        \PY{n}{stdsim} \PY{o}{=} \PY{n}{np}\PY{o}{.}\PY{n}{std}\PY{p}{(}\PY{n}{simdata}\PY{p}{)}
        \PY{n+nb}{print}\PY{p}{(}\PY{n}{meansim}\PY{p}{)}
        \PY{n+nb}{print}\PY{p}{(}\PY{n}{stdsim}\PY{p}{)}
\end{Verbatim}


    \begin{Verbatim}[commandchars=\\\{\}]
2.7638477390381784
0.10477019886868283

    \end{Verbatim}

    \begin{Verbatim}[commandchars=\\\{\}]
{\color{incolor}In [{\color{incolor}8}]:} \PY{n}{height}\PY{p}{,} \PY{n}{bins}\PY{p}{,} \PY{n}{\PYZus{}} \PY{o}{=} \PY{n}{plot}\PY{o}{.}\PY{n}{hist}\PY{p}{(}\PY{n}{simdata}\PY{p}{,}\PY{l+m+mi}{50}\PY{p}{)}
        \PY{n}{x}\PY{o}{=}\PY{n}{np}\PY{o}{.}\PY{n}{linspace}\PY{p}{(}\PY{n}{bins}\PY{p}{[}\PY{l+m+mi}{0}\PY{p}{]}\PY{p}{,}\PY{n}{bins}\PY{p}{[}\PY{o}{\PYZhy{}}\PY{l+m+mi}{1}\PY{p}{]}\PY{p}{,}\PY{l+m+mi}{100}\PY{p}{)}
        \PY{n}{width} \PY{o}{=} \PY{n}{bins}\PY{p}{[}\PY{l+m+mi}{1}\PY{p}{]}\PY{o}{\PYZhy{}}\PY{n}{bins}\PY{p}{[}\PY{l+m+mi}{0}\PY{p}{]}
        \PY{n}{area} \PY{o}{=} \PY{n}{height}\PY{o}{*}\PY{n}{width}
        \PY{n}{amp1} \PY{o}{=} \PY{n}{np}\PY{o}{.}\PY{n}{sum}\PY{p}{(}\PY{n}{area}\PY{p}{)}
        \PY{n}{y}\PY{o}{=}\PY{n}{Gaussian}\PY{p}{(}\PY{n}{amp1}\PY{p}{,}\PY{n}{x}\PY{p}{,}\PY{n}{meansim}\PY{p}{,}\PY{n}{stdsim}\PY{p}{)}
        \PY{n}{plot}\PY{o}{.}\PY{n}{plot}\PY{p}{(}\PY{n}{x}\PY{p}{,}\PY{n}{y}\PY{p}{)}
        \PY{n}{plot}\PY{o}{.}\PY{n}{show}\PY{p}{(}\PY{p}{)}
        \PY{c+c1}{\PYZsh{} distribution of results}
        \PY{c+c1}{\PYZsh{} when bin number increases, the graph becomes closer to the gaussian.}
        \PY{c+c1}{\PYZsh{} but when the number approaches to the total number of data, }
        \PY{c+c1}{\PYZsh{} the graph is small and almost unreadable.}
        \PY{c+c1}{\PYZsh{} when the number decreases, the graph doesnt look like gaussin anymore,}
        \PY{c+c1}{\PYZsh{} with small nunmber (1\PYZhy{}5), the graph doesnt make sens.}
\end{Verbatim}


    \begin{center}
    \adjustimage{max size={0.9\linewidth}{0.9\paperheight}}{output_7_0.png}
    \end{center}
    { \hspace*{\fill} \\}
    
    \begin{Verbatim}[commandchars=\\\{\}]
{\color{incolor}In [{\color{incolor}9}]:} \PY{n+nb}{print}\PY{p}{(}\PY{n}{height}\PY{p}{)}
        \PY{n+nb}{print}\PY{p}{(}\PY{n}{bins}\PY{p}{)}
\end{Verbatim}


    \begin{Verbatim}[commandchars=\\\{\}]
[  3.   0.   1.   3.   3.   2.   3.  16.  15.  32.  40.  55.  66.  89.
 124. 176. 181. 231. 283. 347. 356. 415. 472. 536. 532. 538. 588. 600.
 550. 521. 504. 428. 396. 365. 310. 269. 227. 172. 131. 122.  81.  71.
  50.  39.  15.  17.   7.  12.   1.   5.]
[2.36137462 2.37638749 2.39140036 2.40641323 2.4214261  2.43643897
 2.45145184 2.46646472 2.48147759 2.49649046 2.51150333 2.5265162
 2.54152907 2.55654194 2.57155481 2.58656768 2.60158056 2.61659343
 2.6316063  2.64661917 2.66163204 2.67664491 2.69165778 2.70667065
 2.72168353 2.7366964  2.75170927 2.76672214 2.78173501 2.79674788
 2.81176075 2.82677362 2.8417865  2.85679937 2.87181224 2.88682511
 2.90183798 2.91685085 2.93186372 2.94687659 2.96188946 2.97690234
 2.99191521 3.00692808 3.02194095 3.03695382 3.05196669 3.06697956
 3.08199243 3.09700531 3.11201818]

    \end{Verbatim}

    \begin{Verbatim}[commandchars=\\\{\}]
{\color{incolor}In [{\color{incolor}10}]:} \PY{n}{np}\PY{o}{.}\PY{n}{savetxt}\PY{p}{(}\PY{l+s+s2}{\PYZdq{}}\PY{l+s+s2}{Height}\PY{l+s+s2}{\PYZdq{}}\PY{p}{,} \PY{n}{height}\PY{p}{)}
         \PY{n}{np}\PY{o}{.}\PY{n}{savetxt}\PY{p}{(}\PY{l+s+s2}{\PYZdq{}}\PY{l+s+s2}{Bin}\PY{l+s+s2}{\PYZdq{}}\PY{p}{,}\PY{n}{bins}\PY{p}{)}
\end{Verbatim}


    \begin{Verbatim}[commandchars=\\\{\}]
{\color{incolor}In [{\color{incolor}11}]:} \PY{n}{means} \PY{o}{=} \PY{p}{[}\PY{p}{]}
         \PY{k}{for} \PY{n}{i} \PY{o+ow}{in} \PY{n+nb}{range}\PY{p}{(}\PY{l+m+mi}{500}\PY{p}{)}\PY{p}{:}
             \PY{n}{x} \PY{o}{=} \PY{n}{np}\PY{o}{.}\PY{n}{mean}\PY{p}{(}\PY{n}{sim}\PY{p}{[}\PY{n}{i}\PY{p}{]}\PY{p}{)}
             \PY{n}{means}\PY{o}{.}\PY{n}{append}\PY{p}{(}\PY{n}{x}\PY{p}{)}
\end{Verbatim}


    \begin{Verbatim}[commandchars=\\\{\}]
{\color{incolor}In [{\color{incolor}12}]:} \PY{c+c1}{\PYZsh{} mean and std of the distribution of results}
         \PY{n}{meanmean} \PY{o}{=} \PY{n}{np}\PY{o}{.}\PY{n}{mean}\PY{p}{(}\PY{n}{means}\PY{p}{)}
         \PY{n}{stdmean} \PY{o}{=} \PY{n}{np}\PY{o}{.}\PY{n}{std}\PY{p}{(}\PY{n}{means}\PY{p}{)}
         \PY{n+nb}{print}\PY{p}{(}\PY{n}{meanmean}\PY{p}{)}
         \PY{n+nb}{print}\PY{p}{(}\PY{n}{stdmean}\PY{p}{)}
\end{Verbatim}


    \begin{Verbatim}[commandchars=\\\{\}]
2.7638477390381784
0.023721385157223742

    \end{Verbatim}

    \begin{Verbatim}[commandchars=\\\{\}]
{\color{incolor}In [{\color{incolor}13}]:} \PY{n}{height2}\PY{p}{,} \PY{n}{bins2}\PY{p}{,} \PY{n}{\PYZus{}} \PY{o}{=} \PY{n}{plot}\PY{o}{.}\PY{n}{hist}\PY{p}{(}\PY{n}{means}\PY{p}{,}\PY{l+m+mi}{30}\PY{p}{)}
         \PY{n}{x2}\PY{o}{=}\PY{n}{np}\PY{o}{.}\PY{n}{linspace}\PY{p}{(}\PY{n}{bins2}\PY{p}{[}\PY{l+m+mi}{0}\PY{p}{]}\PY{p}{,}\PY{n}{bins2}\PY{p}{[}\PY{o}{\PYZhy{}}\PY{l+m+mi}{1}\PY{p}{]}\PY{p}{,}\PY{l+m+mi}{100}\PY{p}{)}
         \PY{n}{width2} \PY{o}{=} \PY{n}{bins2}\PY{p}{[}\PY{l+m+mi}{1}\PY{p}{]}\PY{o}{\PYZhy{}}\PY{n}{bins2}\PY{p}{[}\PY{l+m+mi}{0}\PY{p}{]}
         \PY{n}{area2} \PY{o}{=} \PY{n}{height2}\PY{o}{*}\PY{n}{width2}
         \PY{n}{amp2} \PY{o}{=} \PY{n}{np}\PY{o}{.}\PY{n}{sum}\PY{p}{(}\PY{n}{area2}\PY{p}{)}
         \PY{n}{y2}\PY{o}{=}\PY{n}{Gaussian}\PY{p}{(}\PY{n}{amp2}\PY{p}{,}\PY{n}{x2}\PY{p}{,}\PY{n}{meanmean}\PY{p}{,}\PY{n}{stdmean}\PY{p}{)}
         \PY{n}{plot}\PY{o}{.}\PY{n}{plot}\PY{p}{(}\PY{n}{x2}\PY{p}{,}\PY{n}{y2}\PY{p}{)}
         \PY{n}{plot}\PY{o}{.}\PY{n}{show}\PY{p}{(}\PY{p}{)}
         \PY{c+c1}{\PYZsh{} distribution of results}
         \PY{c+c1}{\PYZsh{} same effect as above.}
         \PY{c+c1}{\PYZsh{} when bin number goes up, the graph looks like noises.}
\end{Verbatim}


    \begin{center}
    \adjustimage{max size={0.9\linewidth}{0.9\paperheight}}{output_12_0.png}
    \end{center}
    { \hspace*{\fill} \\}
    
    \begin{Verbatim}[commandchars=\\\{\}]
{\color{incolor}In [{\color{incolor}14}]:} \PY{n}{np}\PY{o}{.}\PY{n}{savetxt}\PY{p}{(}\PY{l+s+s1}{\PYZsq{}}\PY{l+s+s1}{Height2}\PY{l+s+s1}{\PYZsq{}}\PY{p}{,} \PY{n}{height2}\PY{p}{)}
         \PY{n}{np}\PY{o}{.}\PY{n}{savetxt}\PY{p}{(}\PY{l+s+s1}{\PYZsq{}}\PY{l+s+s1}{Bin2}\PY{l+s+s1}{\PYZsq{}}\PY{p}{,} \PY{n}{bins2}\PY{p}{)}
\end{Verbatim}


    \begin{Verbatim}[commandchars=\\\{\}]
{\color{incolor}In [{\color{incolor}57}]:} \PY{n}{factor} \PY{o}{=} \PY{n}{stdmean}\PY{o}{/}\PY{n}{stdsim}
         \PY{n+nb}{print}\PY{p}{(}\PY{n}{factor}\PY{p}{)}
         \PY{n}{formula}\PY{o}{=} \PY{n}{stdsim}\PY{o}{/}\PY{p}{(}\PY{n}{np}\PY{o}{.}\PY{n}{sqrt}\PY{p}{(}\PY{n+nb}{len}\PY{p}{(}\PY{n}{simdata}\PY{p}{)}\PY{p}{)}\PY{p}{)}
         \PY{n+nb}{print}\PY{p}{(}\PY{n}{formula}\PY{p}{)}
         \PY{c+c1}{\PYZsh{} 4. }
         \PY{c+c1}{\PYZsh{} the std of distribution of results is 22.3\PYZpc{} the std of distribution of measurements which means that the std of distribution}
         \PY{c+c1}{\PYZsh{} of the results is smaller then the std of distribution of the measurements by a factor of about 5.}
         
         \PY{n}{stdsim}\PY{o}{/}\PY{n}{np}\PY{o}{.}\PY{n}{sqrt}\PY{p}{(}\PY{l+m+mi}{10000}\PY{p}{)}
         \PY{c+c1}{\PYZsh{} }
\end{Verbatim}


    \begin{Verbatim}[commandchars=\\\{\}]
0.22641347838764456
0.0010477019886868282

    \end{Verbatim}

\begin{Verbatim}[commandchars=\\\{\}]
{\color{outcolor}Out[{\color{outcolor}57}]:} 0.0010477019886868282
\end{Verbatim}
            
    \begin{Verbatim}[commandchars=\\\{\}]
{\color{incolor}In [{\color{incolor}64}]:} \PY{c+c1}{\PYZsh{}5}
         \PY{n}{stderrmean}\PY{o}{=} \PY{n}{stdmean}\PY{o}{/}\PY{p}{(}\PY{n}{np}\PY{o}{.}\PY{n}{sqrt}\PY{p}{(}\PY{n+nb}{len}\PY{p}{(}\PY{n}{means}\PY{p}{)}\PY{p}{)}\PY{p}{)}
         \PY{n+nb}{print}\PY{p}{(}\PY{n}{stderrmean}\PY{p}{)}
\end{Verbatim}


    \begin{Verbatim}[commandchars=\\\{\}]
0.0010608525946401363

    \end{Verbatim}

    \begin{Verbatim}[commandchars=\\\{\}]
{\color{incolor}In [{\color{incolor}16}]:} \PY{n}{height3}\PY{p}{,} \PY{n}{bins3}\PY{p}{,} \PY{n}{\PYZus{}} \PY{o}{=} \PY{n}{plot}\PY{o}{.}\PY{n}{hist}\PY{p}{(}\PY{n}{simdata}\PY{p}{,}\PY{l+m+mi}{6}\PY{p}{)}
         \PY{n}{plot}\PY{o}{.}\PY{n}{show}\PY{p}{(}\PY{p}{)}
\end{Verbatim}


    \begin{center}
    \adjustimage{max size={0.9\linewidth}{0.9\paperheight}}{output_16_0.png}
    \end{center}
    { \hspace*{\fill} \\}
    
    \begin{Verbatim}[commandchars=\\\{\}]
{\color{incolor}In [{\color{incolor}17}]:} \PY{n}{sig1} \PY{o}{=} \PY{p}{(}\PY{n}{height3}\PY{p}{[}\PY{l+m+mi}{2}\PY{p}{]}\PY{o}{+}\PY{n}{height3}\PY{p}{[}\PY{l+m+mi}{3}\PY{p}{]}\PY{p}{)}\PY{o}{/}\PY{n}{np}\PY{o}{.}\PY{n}{sum}\PY{p}{(}\PY{n}{height3}\PY{p}{)}
         \PY{n+nb}{print}\PY{p}{(}\PY{n}{sig1}\PY{p}{)} \PY{c+c1}{\PYZsh{} fraction within 1 sigma}
         \PY{n+nb}{print}\PY{p}{(}\PY{l+m+mf}{34.1}\PY{o}{*}\PY{l+m+mi}{2}\PY{p}{)} \PY{c+c1}{\PYZsh{} fraction of gaussian distribution}
\end{Verbatim}


    \begin{Verbatim}[commandchars=\\\{\}]
0.7496
68.2

    \end{Verbatim}

    \begin{Verbatim}[commandchars=\\\{\}]
{\color{incolor}In [{\color{incolor}18}]:} \PY{n}{sig2} \PY{o}{=} \PY{p}{(}\PY{n}{height3}\PY{p}{[}\PY{l+m+mi}{1}\PY{p}{]}\PY{o}{+}\PY{n}{height3}\PY{p}{[}\PY{l+m+mi}{2}\PY{p}{]}\PY{o}{+}\PY{n}{height3}\PY{p}{[}\PY{l+m+mi}{3}\PY{p}{]}\PY{o}{+}\PY{n}{height3}\PY{p}{[}\PY{l+m+mi}{4}\PY{p}{]}\PY{p}{)}\PY{o}{/}\PY{n}{np}\PY{o}{.}\PY{n}{sum}\PY{p}{(}\PY{n}{height3}\PY{p}{)}
         \PY{n+nb}{print}\PY{p}{(}\PY{n}{sig2}\PY{p}{)} \PY{c+c1}{\PYZsh{} fraction within 2 sigma}
         \PY{n+nb}{print}\PY{p}{(}\PY{l+m+mf}{34.1}\PY{o}{*}\PY{l+m+mi}{2}\PY{o}{+}\PY{l+m+mf}{13.6}\PY{o}{*}\PY{l+m+mi}{2}\PY{p}{)} \PY{c+c1}{\PYZsh{} fraction of gaussian distribution}
\end{Verbatim}


    \begin{Verbatim}[commandchars=\\\{\}]
0.9799
95.4

    \end{Verbatim}

    \begin{Verbatim}[commandchars=\\\{\}]
{\color{incolor}In [{\color{incolor}19}]:} \PY{n}{sig3} \PY{o}{=} \PY{p}{(}\PY{n}{height3}\PY{p}{[}\PY{l+m+mi}{0}\PY{p}{]}\PY{o}{+}\PY{n}{height3}\PY{p}{[}\PY{l+m+mi}{1}\PY{p}{]}\PY{o}{+}\PY{n}{height3}\PY{p}{[}\PY{l+m+mi}{2}\PY{p}{]}\PY{o}{+}\PY{n}{height3}\PY{p}{[}\PY{l+m+mi}{3}\PY{p}{]}\PY{o}{+}\PY{n}{height3}\PY{p}{[}\PY{l+m+mi}{4}\PY{p}{]}\PY{o}{+}\PY{n}{height3}\PY{p}{[}\PY{l+m+mi}{5}\PY{p}{]}\PY{p}{)}\PY{o}{/}\PY{n}{np}\PY{o}{.}\PY{n}{sum}\PY{p}{(}\PY{n}{height3}\PY{p}{)}
         \PY{n+nb}{print}\PY{p}{(}\PY{n}{sig3}\PY{p}{)} \PY{c+c1}{\PYZsh{} fraction within 3 sigma}
         \PY{n+nb}{print}\PY{p}{(}\PY{l+m+mf}{34.1}\PY{o}{*}\PY{l+m+mi}{2}\PY{o}{+}\PY{l+m+mf}{13.6}\PY{o}{*}\PY{l+m+mi}{2}\PY{o}{+}\PY{l+m+mf}{2.1}\PY{o}{*}\PY{l+m+mi}{2}\PY{p}{)} \PY{c+c1}{\PYZsh{} fraction of gaussian distribution}
\end{Verbatim}


    \begin{Verbatim}[commandchars=\\\{\}]
1.0
99.60000000000001

    \end{Verbatim}

    \begin{Verbatim}[commandchars=\\\{\}]
{\color{incolor}In [{\color{incolor}20}]:} \PY{c+c1}{\PYZsh{} we can see that the fractions are similar except that the simulated data are more concentrated within 1 sigma than the gaussian}
\end{Verbatim}



    % Add a bibliography block to the postdoc
    
    
    
    \end{document}
